\chapter{KESIMPULAN DAN SARAN}
Pada bagian ini dijelaskan mengenai kesimpulan dari penelitian yang telah dilakukan. Penjelasan dibagi menjadi beberapa bagian, yaitu kesimpulan, dan saran.

\section{Kesimpulan}
Kesimpulan dari penelitian ini adalah sebagai berikut:
\begin{enumerate}
    \item Sistem autentikasi M2M berbasis risiko menggunakan Random Forest dapat mengklasikasi risiko autentikasi. Dengan akurasi 70.8\%, presisi 70.1\%, \textit{recall} 96.8\%, dan \textit{F1-score} 71.3\%.
    \item Berdasarkan hasil pengujian, kinerja sistem komputasi dan waktu meningkat seiring dengan bertambahnya ukuran dataset. Dan relative stabil seiring dengan bertambahnya ukuran dataset.
    \item Pembatasan fitur kepentingan dapat berpengaruh pada akurasi sistem.
\end{enumerate}


\section{Saran}
Penelitian ini masih memiliki beberapa kekurangan yang dapat diperbaiki pada penelitian selanjutnya, yaitu:
\begin{enumerate}
    \item Penelitian ini masih menggunakan dataset sintetis. Sehingga perlu dilakukan penelitian lebih lanjut dengan menggunakan dataset asli.
    \item Akurasi sistem masih dapat ditingkatkan, serta perlu dilakukan penelitian lebih lanjut untuk meningkatkan keamanan sistem.
    \item Opitimasi parameter Random Forest masih dapat dilakukan lebih lanjut.
\end{enumerate}
