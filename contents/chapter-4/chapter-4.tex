\chapter{Analisis dan Rancangan Sistem}

\section{Analisis Sistem}

Analisis sistem terdiri dari gambaran umum sistem yang dapat dilihat pada baguan 4.1.1 dan analisis kebutuhan sistem yang dapat dilihat pada bagian 4.1.2.

\subsection{Gambaran Umum Sistem}
\begin{figure}[H]
    \centering
    \includegraphics[width=0.6\textwidth]{contents/chapter-4/gambaran-umum.png}
    \caption{Gambaran Umum Sistem}
    \label{fig:gambaran-umum}
\end{figure}
Gambar \ref{fig:gambaran-umum} menjelaskan secara umum system bekerja dengan menggunakan metadata login sebagai input untuk mengidentifikasi risiko dari suatu transaksi. Metadata login ini kemudian diolah dan dianalisis menggunakan metode Random Forest untuk menghasilkan prediksi risiko autentikasi dengan output nya adalah klasifikasi.

Data login menggunakan dataset yang disintesis dari lebih dari 33 juta upaya login dan sekitar 3,3 juta pengguna pada layanan online skala besar di Norwegia. Data asli dikumpulkan antara Februari 2020 dan Februari 2021.

Setelah itu dataset akan digunakan untuk proses pembuatan model klasi random forest, proses pembuatan model akan melewati tiga tahap yaitu : \textit{preprocessing}, \textit{train model}
dan prediksi dataset


\subsection{Analisis Kebutuhan Sistem}
Dalam membangun sistem ini, diperlukan analisa kebutuhan fungsional dan non-fungsional. Kebutuhan fungsional adalah kebutuhan yang berkaitan dengan fungsi-fungsi yang harus ada dalam sistem. Kebutuhan non-fungsional adalah kebutuhan yang berkaitan dengan kualitas sistem yang dibangun. Kebutuhan fungsional dan non-fungsional dapat dilihat pada bagian 4.1 dan Tabel 4.2.

\subsection{Kebutuhan Fungsional}
Kebutuhan fungsional sistem ini adalah sebagai berikut:
\begin{enumerate}
    \item Sistem dapat melakukan analisis risiko autentikasi dengan menggunakan metode Random Forest.
    \item Sistem dapat men-genrate token autentikasi dari input user id.
    \item Sistem risiko autentikasi dapat terintegrasi dengan sistem FHIR.
\end{enumerate}

\subsection{Kebutuhan Non-Fungsional}
Analisis kebutuhan non fungsional dibagi menjadi dua, yaitu analisis
kebutuhan perangkat lunak dan analisis kebutuhan perangkat keras. Analisis
perangkat keras bertujuan untuk memudahkan proses perancangan dan
implementasi dalam pembangunan sistem ini.

\subsubsection{Analisis Kebutuhan Perangkat Lunak}

\begin{enumerate}
    \item Bahasa pemrograman python versi 3.9 dengan anaconda 3
    \item Framework flask untuk membuat endpoint 
    \item Sistem operasi dengan base unix untuk menjalankan sistem klasifikasi
\end{enumerate}

\subsubsection{Analisis Kebutuhan Perangkat Keras}
\begin{enumerate}
    \item Laptop atau PC dengan RAM minimal 8gb
    \item Prosessor dengan minimum 5 CPU Core 
    \item \textit{Storage} dengan minimum 50gb
\end{enumerate}

\section{Rancangan Sistem}
Berikut adalah rancangan sistem yang akan dibangun. Rancangan sistem terdiri dari rancangan arsitektur sistem, rancangan pembersihan data, rancangan variabel kepentingan, dan rancangan integrasi dengan sistem FHIR.

\subsection{Rancangan Arsitektur Sistem}
Rancangan arsitektur sistem dapat dilihat pada Gambar \ref{fig:arsitektur-sistem}. Sistem ini terdiri dari 3 komponen utama yaitu komponen \textit{data preprocessing}, komponen \textit{data mining}, dan komponen \textit{data integration}. Komponen \textit{data preprocessing} berfungsi untuk membersihkan data dari \textit{noise} dan \textit{outlier}. Komponen \textit{data mining} berfungsi untuk melakukan analisis risiko autentikasi dengan menggunakan metode Random Forest. Komponen \textit{data integration} berfungsi untuk mengintegrasikan sistem dengan sistem FHIR.

\begin{figure}[H]
    \centering
    \includegraphics[width=0.5\textwidth]{contents/chapter-4/diagram-khusus.png}
    \caption{Rancangan Arsitektur Sistem}
    \label{fig:arsitektur-sistem}
\end{figure}

Pada Gambar \ref{fig:arsitektur-sistem}, sistem ini akan mendapatkan data login dari datasest. Data login ini kemudian akan digunakan sebagai input untuk melakukan analisis risiko autentikasi.

Dalam menentukan risiko autentikasi, sistem ini akan menggunakan metode Random Forest. Metode Random Forest akan menghasilkan variabel kepentingan yang dapat digunakan untuk melakukan analisis risiko autentikasi.

Setelah itu, sistem ini akan terintegrasi dengan sistem FHIR. Sistem ini akan menggunakan FHIR API untuk mengakses data dari sistem FHIR. FHIR API akan mengakses data dari sistem FHIR dengan menggunakan \textit{request} dan \textit{response}.

\subsection{Rancangan Pembersihan Data}
Rancangan pembersihan data dapat dilihat pada Gambar \ref{fig:pembersihan-data} Pada tahap ini, data akan dibersihkan dari \textit{noise} dan \textit{outlier}. \textit{Noise} adalah data yang tidak memiliki nilai yang berarti. \textit{Outlier} adalah data yang memiliki nilai yang ekstrim. Pada tahap ini, data akan dibersihkan dari \textit{noise} dan \textit{outlier} dengan menggunakan beberapa metode yaitu :

\begin{figure}[H]
    \centering
    \includegraphics[width=0.4\textwidth]{contents/chapter-4/pre-processing.png}
    \caption{Rancangan Pembersihan Data}
    \label{fig:pembersihan-data}
\end{figure}

Dalam melakukan pembersihan data, sistem ini akan satu metode yaitu :

\begin{enumerate}
    \item \textit{Missing Value} : Menghapus data yang memiliki nilai kosong.
\end{enumerate}

Setelah data dibersihkan, data akan digunakan sebagai input untuk melakukan analisis risiko autentikasi. Data ini kemudian akan digunakan sebagai input untuk melakukan analisis risiko autentikasi.

\subsection{Rancangan Variabel Kepentingan}
Rancangan variabel kepentingan akan dilakukan dengan menggunakan metode Random Forest. Metode Random Forest akan menghasilkan variabel kepentingan yang dapat dilihat pada Gambar \ref{fig:variabel-kepentingan}. Variabel kepentingan ini akan digunakan untuk melakukan analisis risiko autentikasi.
Berikut adalah rancangan variabel kepentingan yang akan digunakan untuk melakukan analisis risiko autentikasi.
\begin{figure}[H]
    \centering
    \includegraphics[width=0.2\textwidth]{contents/chapter-4/vim.drawio.png}
    \caption{Rancangan Variabel Kepentingan}
    \label{fig:variabel-kepentingan}
\end{figure}

Gambar \ref{fig:variabel-kepentingan} menjelaskan bahwa variabel kepentingan akan digunakan untuk melakukan analisis risiko autentikasi. Variabel kepentingan ini akan digunakan sebagai input untuk melakukan analisis risiko autentikasi.

\subsection{Rancangan Integrasi Dengan Sistem FHIR}
Rancangan integrasi dengan sistem FHIR dapat dilihat pada Gambar 4.6. Sistem ini akan terintegrasi dengan sistem FHIR untuk mendapatkan data login dari pasien. Data login ini kemudian akan digunakan sebagai input untuk melakukan analisis risiko autentikasi.
\begin{figure}[H]
    \centering
    \includegraphics[width=0.2\textwidth]{contents/chapter-4/fhir-rba.drawio.png}
    \caption{Rancangan Integrasi Dengan Sistem FHIR}
    \label{fig:integrasi}
\end{figure}

Untuk melakukan integrasi dengan sistem FHIR, sistem ini akan menggunakan FHIR API. FHIR API adalah sebuah API yang digunakan untuk mengakses data dari sistem FHIR. FHIR API akan mengakses data dari sistem FHIR dengan menggunakan \textit{request} dan \textit{response}.

\section{Rancangan Pengujian}
Pengujian sistem ini akan dilakukan dengan menggunakan beberapa metode yaitu:
\begin{enumerate}
    \item Pengujian Fungsional : Pengujian fungsional dilakukan untuk menguji apakah sistem dapat berjalan dengan baik sesuai dengan kebutuhan fungsional yang telah ditentukan.
    \item Menentukan Evaluasi : Akurasi, Presisi, \textit{Recall}, \textit{F1 Score}, dan \textit{Confusion Matrix} akan digunakan untuk menentukan evaluasi dari sistem.
\end{enumerate}
