\chapter{Tinjauan Pustaka}

Autentikasi berbasis risiko (RBA) adalah metode untuk memverifikasi identitas
pengguna dengan menyesuaikan tingkat autentikasi secara dinamis berdasarkan
tingkat risiko sesi saat ini. Pendekatan ini bertujuan untuk menyeimbangkan
keamanan dan kenyamanan dengan menyediakan langkah-langkah autentikasi yang
lebih kuat ketika tingkat risiko tinggi, dan langkah-langkah yang lebih longgar
ketika tingkat risiko rendah.

Sebuah tinjauan literatur mengenai Autentikasi Berbasis Risiko menemukan
bahwa banyak penelitian telah dilakukan pada topik ini dan berbagai teknik telah
diusulkan. Salah satu teknik yang paling umum adalah menggunakan algoritma
penilaian risiko untuk secara dinamis menyesuaikan tingkat otentikasi berdasarkan
tingkat risiko.

Studi yang dilakukan oleh Thomas dkk. (2017) membahas resiko dari password
yang dicuri dan bagaimana kebocoran kredensial dapat terjadi. Tidak hanya itu
namun studi tersebut juga menampilkan situs situs yang banyak mengalami
kebocoran data. Resiko yang paling besar dapat terjadi adalah data-data kita
disalahgunakan hingga mengalami kerugian material. Sedangkan phising menjadi
faktor utama penyebab terjadinya kebocoran kredensial dan disusul oleh
keyloggers.

Stephan Wiefling dkk. (2022) mengemukakan Risk-Based Authentication
(RBA) dapat memperkirakan apakah login itu sah atau merupakan upaya
pengambilalihan akun. Ini dilakukan dengan memantau dan merekam sekumpulan
fitur yang tersedia dalam konteks login. Fitur potensial berkisar dari jaringan (mis.,
alamat IP), perangkat atau klien (mis., string agen pengguna), hingga informasi
biometrik perilaku (mis., waktu masuk).Selain itu kelebihan RBA juga telah disurvey oleh Cabarcos dkk. (2019)
menganalisis literatur tentang autentikasi adaptif berdasarkan prinsip-prinsip desain
yang terkenal dalam disiplin sistem berbasis resiko dan tantangan nya adalah tidak
ada satu ukuran yang cocok untuk semua dalam keamanan, tidak ada mekanisme baru yang akan menggantikan semua mekanisme lainnya dan diterima sebagai
solusi universal. Doerfler dkk. (2019) menggambarkan bahwa tantangan login
bertindak sebagai penghalang penting untuk pembajakan, tetapi gesekan dalam
proses menyebabkan pengguna yang sah gagal masuk, meskipun pada akhirnya
dapat mengakses akun mereka lagi.

Banyak sistem yang sudah mengimplementasikan RBA karena kelebihannya,
studi yang dilakukan oleh Prasad dkk. (2017) menjadi awal mula bagaimana sistem
perbankan mulai menerapkan autentikasi berdasarkan risiko dengan kombinasi
lokasi. Sedangkan dalam sektor kesehatan sendiri autentikasi standar seperti user
dan password masih banyak digunakan, karena sistem IT kesehatan masih fokus
dalam mengembangkan The Fast Health Interoperability Resources (FHIR) Ayaz
dkk. (2021)

Selanjutnya, beberapa studi dalam literatur mengusulkan metode otentikasi
berbasis risiko yang menggunakan berbagai faktor seperti lokasi, waktu, dan jenis
perangkat untuk menentukan tingkat risiko suatu sesi. Sebagai contoh, sebuah
penelitian oleh Agarwal dkk. (2016) mengusulkan sistem RBA berbasis lokasi yang
menggunakan lokasi perangkat pengguna untuk menentukan tingkat risiko suatu
sesi. Studi ini menemukan bahwa sistem yang diusulkan secara efektif
meningkatkan keamanan sistem dengan tetap mempertahankan kegunaan.

Penggunaan RBA masih terbatas pada major digital service, hal ini sebagian
disebabkan oleh kurangnya pengetahuan dan implementasi terbuka yang
memungkinkan penyedia layanan mana pun untuk meluncurkan perlindungan RBA
kepada penggunanya. Untuk menutup kesenjangan ini, Stephan Wiefling dkk.
(2021) memberikan analisis tentang karakteristik RBA dalam penerapan praktis
sekaligus memberikan dataset yang dapat digunakan secara umum.
Penelitian lain Misbahuddin dkk. (2017) mengusulkan sistem RBA berbasis
perangkat yang menggunakan jenis perangkat dan status perangkat untuk
menentukan tingkat risiko suatu sesi. Penelitian tersebut menemukan bahwa sistem
yang diusulkan secara efektif meningkatkan keamanan sistem dengan tetap
mempertahankan kegunaan menggunakan machine learning.

Penggunaan analisis berbasis risiko dalam konteks machine to machine dibahas
dalam studi yang dilakukan oleh Taneja (2013). Mekanisme keamanan tertentu
mengasumsikan bahwa akhir perangkat sudah diamankan. Dalam jaringan IoT,
perangkat IoT itu sendiri dapat dikompromikan. Seorang penyerang dapat mencuri
perangkat, mendapatkan akses mengaksesnya dan menggunakannya untuk
serangan yang lebih merusak.

Roy dan Dasgupta (2018) sudah meneliti bahwa fuzzy dapat menjadi terobosan
dalam menentukan multifaktor autentikasi. Selain itu, banyak penelitian juga telah
mengusulkan penggunaan algoritma pembelajaran mesin seperti pohon keputusan,
Random Forest, dan jaringan syaraf untuk meningkatkan kinerja RBA. Sebagai
contoh, sebuah penelitian oleh Zhang, F dkk. (2012) mengusulkan sistem RBA
yang menggunakan algoritma Random Forest untuk menentukan tingkat risiko dari
sebuah sesi. Penelitian ini menemukan bahwa sistem yang diusulkan mencapai
tingkat akurasi yang tinggi dan meningkatkan keamanan sistem. Dalam studi lain
Alam dan Vuong (2013), Speiser dkk. (2019) menunjukkan bahwa Random Forest
adalah pilihan yang baik karena dapat secara efektif mengklasifikasikan transaksi
berdasarkan tingkat resikonya menggunakan serangkaian fitur yang berasal dari
data transaksi. Random Forest adalah algoritma pembelajaran mesin yang kuat yang
dapat menangani kumpulan data besar dan mampu menangani kebisingan dan nilai
yang hilang dengan baik. Selain itu, dapat memberikan skor kepentingan fitur, yang
dapat digunakan untuk mengidentifikasi fitur yang paling penting untuk klasifikasi
risiko. Secara keseluruhan, Random Forest adalah algoritma pembelajaran mesin
yang efektif dan banyak digunakan untuk otentikasi M2M berbasis risiko.

Rangkuman penelitian sebelumnya dapat dilihat pada Tabel 2.1. Dalam studi
ini ditawarkan pendekatan autentikasi berbasis risiko dengan menggunakan dalam
kasus machine to machine device yang dikaitkan dalam FHIR service.

\begin{longtable}{|p{1.7cm}|p{3.8cm}|p{3.5cm}|p{5cm}|}
    \caption{Tinjauan Pustaka}\\
    \hline
    \textbf{Nama} & \textbf{Penelitian} & \textbf{Metode} & \textbf{Hasil} \\
    \hline
    \endfirsthead
    \multicolumn{4}{c}{{\bfseries Table \thetable:} Lanjutan Tinjauan Pustaka} \\
    \hline
    \textbf{Nama} & \textbf{Penelitian} & \textbf{Metode} & \textbf{Hasil} \\
    \hline
    \endhead
    \hline
    \multicolumn{4}{r}{{Berlanjut di halaman selanjutnya}} \\
    \endfoot
    \hline
    \endlastfoot
    Thomas dkk (2017) & Pencurian kredensial dan menilai risiko yang ditimbulkannya bagi jutaan pengguna & Framework otomatis yang menggabungkan data Google Search dan Gmail untuk mengidentifikasi lebih dari satu miliar korban kebocoran kredensial, kit phising, dan keylogger. & Mengidentifikasi 788.000 calon korban keylogger siap pakai; 12,4 juta calon korban kit phishing; 1,9 miliar nama pengguna dan kata sandi yang terungkap melalui pelanggaran data dan diperdagangkan di forum pasar gelap. \\
    \hline
    Stephan Wiefling dkk (2022) & Analisis RBA pada layanan online skala besar dunia nyata & Simple model, extended model, login dataset & RBA memblokir 99,5\% penyerang naif. Simple model: targeted attackers dropped dari 0.9552 menjadi 0.5295. \\
    \hline
    Cabarcos dkk (2019) & Survey studi mengenai cara dinamis memilih mekanisme terbaik untuk mengautentikasi pengguna tergantung pada beberapa faktor & CARS-AD (Vector Space Model (VSM)), ASSO ( SVM), Reinforced AuthN (Logistic Regresion) & Pengurangan overhead kata sandi (masing-masing 42\% dan 47\% lebih sedikit permintaan kata sandi). \\
    \hline
    Doerfler dkk (2019) & Manfaat fitur login keamanan untuk mencegah pengambilalihan akun & MFA & Memblokir lebih dari 94\% upaya pembajakan. \\
    \hline
    Prasad dkk (2017) & Meningkatkan Layanan Mobile Banking menggunakan Otentikasi Berbasis Lokasi & GPS dan GPRS & GPS digunakan untuk menyediakan autentikasi lokasi, banyak informasi terkait satelit yang tidak mudah diimplementasikan. \\
    \hline
    Agarwal dkk (2016) & Mengevaluasi strategi autentikasi ulang untuk ponsel & Implicit authentication, Context-aware authentication, App-specific authentication & Dalam hal kinerja tugas, konfigurasi yang diusulkan bekerja sebaik konfigurasi default, namun konfigurasi yang diusulkan dianggap lebih nyaman dan tidak terlalu mengganggu oleh pengguna. \\
    \hline
    Stephan Wiefling dkk (2021) & Memperkuat otentikasi berbasis kata sandi menggunakan Otentikasi berbasis risiko (RBA) & simple model (SIMPLE), extended model (EXTEND), Data e-learning website untuk mahasiswa kedokteran & RBA dapat mencapai tingkat autentikasi ulang yang rendah untuk pengguna yang sah saat memblokir lebih dari 99,45\% serangan yang ditargetkan dengan model EXTEND. \\
    \hline
    Misbahuddin dkk (2017) & Desain sistem otentikasi berbasis risiko menggunakan machine learning & Profile analysis block, Risk Engine, Adaptive Authentication Block, SVM & Teknik yang diajukan menawarkan tiga pilihan untuk risk engine, sehingga dapat beroperasi dalam situasi yang berbeda. \\
    \hline
    Taneja dkk (2013) & Mendeteksi perangkat IoT (M2M) yang disusupi menggunakan perilaku mobilitas & Wireless gateway checking & Metode ini mendeteksi perangkat yang disusupi untuk skenario dimana perilaku device telah berubah. \\
    \hline
    Dasgupta dkk (2018) & Multifactor authentication menggunakan fuzzy decision support system & fuzzy, genetic algorithm & Perbandingan akurasi dengan metode lain: FIDO 89\%, Microsoft Azure 92\%, Adaptive MFA 95\%. \\
    \hline
    Zhang dkk (2012) & Authentikasi dan otorisasi berdasarkan lokasi & Spoofing on the hardware level (GPS), Spoofing on the OS level, Spoofing on the application level (IP, MAC) & Mekanisme autentikasi dan otorisasi berbasis lokasi menjadi lebih aman dan valid. \\
    \hline
    Alam dkk (2013) & Mendeteksi malware pada Android dengan random forest & Random forest, dataset antimalware & 99,9 persen sampel benar. \\
    \hline
    Speicher dkk (2019) & Perbandingan metode pemilihan variabel random forest untuk pemodelan prediksi klasifikasi & Random forest, kondisional random forest & Standar random forest memiliki waktu komputasi dan error rate yang lebih baik dibandingkan dengan kondisional random forest. \\
    \hline
\end{longtable}

