\chapter{Tinjauan Pustaka}

Autentikasi berbasis risiko (RBA) adalah metode untuk memverifikasi identitas
pengguna dengan menyesuaikan tingkat autentikasi secara dinamis berdasarkan
tingkat risiko sesi saat ini. Pendekatan ini bertujuan untuk menyeimbangkan
keamanan dan kenyamanan dengan menyediakan langkah-langkah autentikasi yang
lebih kuat ketika tingkat risiko tinggi, dan langkah-langkah yang lebih longgar
ketika tingkat risiko rendah.

Sebuah tinjauan literatur mengenai Autentikasi Berbasis Risiko menemukan
bahwa banyak penelitian telah dilakukan pada topik ini dan berbagai teknik telah
diusulkan. Salah satu teknik yang paling umum adalah menggunakan algoritma
penilaian risiko untuk secara dinamis menyesuaikan tingkat otentikasi berdasarkan
tingkat risiko.

Studi yang dilakukan oleh Thomas dkk. (2017) membahas resiko dari password
yang dicuri dan bagaimana kebocoran kredensial dapat terjadi. Tidak hanya itu
namun studi tersebut juga menampilkan situs situs yang banyak mengalami
kebocoran data. Resiko yang paling besar dapat terjadi adalah data-data kita
disalahgunakan hingga mengalami kerugian material. Sedangkan phising menjadi
faktor utama penyebab terjadinya kebocoran kredensial dan disusul oleh
keyloggers.

Stephan Wiefling dkk. (2022) mengemukakan Risk-Based Authentication
(RBA) dapat memperkirakan apakah login itu sah atau merupakan upaya
pengambilalihan akun. Ini dilakukan dengan memantau dan merekam sekumpulan
fitur yang tersedia dalam konteks login. Fitur potensial berkisar dari jaringan (mis.,
alamat IP), perangkat atau klien (mis., string agen pengguna), hingga informasi
biometrik perilaku (mis., waktu masuk).Selain itu kelebihan RBA juga telah disurvey oleh Cabarcos dkk. (2019)
menganalisis literatur tentang autentikasi adaptif berdasarkan prinsip-prinsip desain
yang terkenal dalam disiplin sistem berbasis resiko dan tantangan nya adalah tidak
ada satu ukuran yang cocok untuk semua dalam keamanan, tidak ada mekanisme baru yang akan menggantikan semua mekanisme lainnya dan diterima sebagai
solusi universal. Doerfler dkk. (2019) menggambarkan bahwa tantangan login
bertindak sebagai penghalang penting untuk pembajakan, tetapi gesekan dalam
proses menyebabkan pengguna yang sah gagal masuk, meskipun pada akhirnya
dapat mengakses akun mereka lagi.

Banyak sistem yang sudah mengimplementasikan RBA karena kelebihannya,
studi yang dilakukan oleh Prasad dkk. (2017) menjadi awal mula bagaimana sistem
perbankan mulai menerapkan autentikasi berdasarkan risiko dengan kombinasi
lokasi. Sedangkan dalam sektor kesehatan sendiri autentikasi standar seperti user
dan password masih banyak digunakan, karena sistem IT kesehatan masih fokus
dalam mengembangkan The Fast Health Interoperability Resources (FHIR) Ayaz
dkk. (2021)

Selanjutnya, beberapa studi dalam literatur mengusulkan metode otentikasi
berbasis risiko yang menggunakan berbagai faktor seperti lokasi, waktu, dan jenis
perangkat untuk menentukan tingkat risiko suatu sesi. Sebagai contoh, sebuah
penelitian oleh Agarwal dkk. (2016) mengusulkan sistem RBA berbasis lokasi yang
menggunakan lokasi perangkat pengguna untuk menentukan tingkat risiko suatu
sesi. Studi ini menemukan bahwa sistem yang diusulkan secara efektif
meningkatkan keamanan sistem dengan tetap mempertahankan kegunaan.

Penggunaan RBA masih terbatas pada major digital service, hal ini sebagian
disebabkan oleh kurangnya pengetahuan dan implementasi terbuka yang
memungkinkan penyedia layanan mana pun untuk meluncurkan perlindungan RBA
kepada penggunanya. Untuk menutup kesenjangan ini, Stephan Wiefling dkk.
(2021) memberikan analisis tentang karakteristik RBA dalam penerapan praktis
sekaligus memberikan dataset yang dapat digunakan secara umum.
Penelitian lain Misbahuddin dkk. (2017) mengusulkan sistem RBA berbasis
perangkat yang menggunakan jenis perangkat dan status perangkat untuk
menentukan tingkat risiko suatu sesi. Penelitian tersebut menemukan bahwa sistem
yang diusulkan secara efektif meningkatkan keamanan sistem dengan tetap
mempertahankan kegunaan menggunakan machine learning.

Penggunaan analisis berbasis risiko dalam konteks machine to machine dibahas
dalam studi yang dilakukan oleh Taneja (2013). Mekanisme keamanan tertentu
mengasumsikan bahwa akhir perangkat sudah diamankan. Dalam jaringan IoT,
perangkat IoT itu sendiri dapat dikompromikan. Seorang penyerang dapat mencuri
perangkat, mendapatkan akses mengaksesnya dan menggunakannya untuk
serangan yang lebih merusak.

Roy dan Dasgupta (2018) sudah meneliti bahwa fuzzy dapat menjadi terobosan
dalam menentukan multifaktor autentikasi. Selain itu, banyak penelitian juga telah
mengusulkan penggunaan algoritma pembelajaran mesin seperti pohon keputusan,
Random Forest, dan jaringan syaraf untuk meningkatkan kinerja RBA. Sebagai
contoh, sebuah penelitian oleh Zhang, F dkk. (2012) mengusulkan sistem RBA
yang menggunakan algoritma Random Forest untuk menentukan tingkat risiko dari
sebuah sesi. Penelitian ini menemukan bahwa sistem yang diusulkan mencapai
tingkat akurasi yang tinggi dan meningkatkan keamanan sistem. Dalam studi lain
Alam dan Vuong (2013), Speiser dkk. (2019) menunjukkan bahwa Random Forest
adalah pilihan yang baik karena dapat secara efektif mengklasifikasikan transaksi
berdasarkan tingkat resikonya menggunakan serangkaian fitur yang berasal dari
data transaksi. Random Forest adalah algoritma pembelajaran mesin yang kuat yang
dapat menangani kumpulan data besar dan mampu menangani kebisingan dan nilai
yang hilang dengan baik. Selain itu, dapat memberikan skor kepentingan fitur, yang
dapat digunakan untuk mengidentifikasi fitur yang paling penting untuk klasifikasi
risiko. Secara keseluruhan, Random Forest adalah algoritma pembelajaran mesin
yang efektif dan banyak digunakan untuk otentikasi M2M berbasis risiko.

Rangkuman penelitian sebelumnya dapat dilihat pada Tabel 2.1. Dalam studi
ini ditawarkan pendekatan autentikasi berbasis risiko dengan menggunakan dalam
kasus machine to machine device yang dikaitkan dalam FHIR service.

% todo: tabel 2.1

