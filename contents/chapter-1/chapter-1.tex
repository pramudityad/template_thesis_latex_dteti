\chapter{PENDAHULUAN}

\section{Latar Belakang}

Dalam sistem kesehatan digital, FHIR \textit{(Fast Healthcare Interoperability
Resources)} telah menjadi standar yang umum digunakan untuk berbagi data medis
antar sistem. Autentikasi mesin ke mesin (M2M) digunakan untuk mengamankan
akses ke data FHIR oleh aplikasi kesehatan dan sistem lainnya. Namun, metode
autentikasi M2M saat ini cenderung kurang adaptif terhadap risiko keamanan yang
berbeda-beda pada setiap transaksi. Hal ini dapat menyebabkan celah keamanan
dan penyalahgunaan data medis oleh pihak yang tidak berwenang.

Untuk mengatasi masalah ini, penelitian sebelumnya telah mengusulkan
penggunaan autentikasi M2M berbasis risiko pada aplikasi online. Namun,
kebanyakan penelitian hanya menggunakan model statistik sederhana atau aturan
heuristik untuk membangun sistem autentikasi M2M berbasis risiko, seperti yang
dilakukan \cite{steinegger_risk-based_2016}. 
Hal ini dapat membatasi kemampuan sistem untuk mengenali ancaman keamanan yang kompleks.

Oleh karena itu, dalam penelitian ini, kami mengusulkan penggunaan metode
machine learning, khususnya Random Forest, untuk membangun sistem autentikasi
M2M berbasis risiko pada sistem FHIR. Dalam penelitian ini, kami akan
membandingkan kinerja sistem autentikasi M2M berbasis risiko menggunakan
Random Forest dengan kondisi sekarang. Kami juga akan mengevaluasi efektivitas
dan efisiensi dari sistem autentikasi M2M berbasis risiko yang diusulkan. Dengan
demikian, penelitian ini diharapkan dapat meningkatkan keamanan dan keandalan
sistem autentikasi M2M pada sistem kesehatan digital berbasis FHIR.


\section{Rumusan Masalah}

Aturan heuristik untuk membangun sistem autentikasi dinilai membatasi
kemampuan sistem untuk mengenali ancaman keamanan yang kompleks seperti \textit{token reply.}


\section{Batasan Masalah}
Agar penelitian ini dapat dilakukan dengan baik, maka perlu dibuat batasan masalah.
Batasan masalah pada penelitian ini adalah:
\begin{enumerate}
	\item Penelitian ini fokus pada mekanisme autentikasi M2M \textit{machine to machine} pada sistem FHIR.
	\item Dataset yang digunakan adalah dataset sintetis login M2M yang dibuat oleh \cite{steinegger_risk-based_2016}
	\item Pemilihan fitur dan dataset akan dibatasi dimaksudkan untuk mengurangi kompleksitas model dan keterbatasan sumber daya komputasi.
\end{enumerate}

\section{Tujuan Penelitian}
Tujuan dari penelitian ini adalah:
\begin{enumerate}
	\item Membangun sistem autentikasi M2M berbasis risiko menggunakan Random Forest.
	\item Mengevaluasi kinerja sistem autentikasi M2M berbasis risiko menggunakan Random Forest.
	\item Mengevaluasi efektivitas dan efisiensi sistem autentikasi M2M berbasis risiko menggunakan Random Forest.
	\item Meningkatkan keamanan dan keandalan sistem autentikasi M2M pada sistem kesehatan digital berbasis FHIR.
\end{enumerate}


\section{Manfaat Penelitian}
Manfaat dari penelitian ini adalah diharapkan sebagai berikut:
\begin{enumerate}
	\item Meningkatkan keamanan dan keandalan sistem autentikasi M2M pada sistem kesehatan digital berbasis FHIR.
	\item Menambah pengetahuan dan wawasan mengenai autentikasi M2M berbasis risiko menggunakan Random Forest.
	\item Meminimalisir risiko keamanan pada sistem kesehatan digital berbasis FHIR.
	\item Dapat memodelkan masalah keamanan dengan menggunakan metode machine learning.
\end{enumerate}
