\chapter{HASIL DAN PEMBAHASAN}
Pada bagian ini dijelaskan mengenai hasil dari penelitian yang telah dilakukan. Penjelasan dibagi menjadi beberapa bagian, yaitu hasil pengujian, analisis hasil pengujian, dan pembahasan hasil pengujian.

\section{Hasil Pengujian}
Hasil pengujian berupa hasil pengujian fungsional, hasil pengujian non-fungsional, hasil pengujian eksperimental, dan hasil pengujian lainnya. Hasil pengujian fungsional berisi hasil pengujian terhadap fitur-fitur yang ada pada sistem. Hasil pengujian non-fungsional berisi hasil pengujian terhadap aspek non-fungsional yang ada pada sistem. Hasil pengujian eksperimental berisi hasil pengujian terhadap sistem yang dibandingkan dengan sistem lainnya. Hasil pengujian lainnya berisi hasil pengujian yang tidak termasuk dalam hasil pengujian fungsional, non-fungsional, dan eksperimental.

\section{Analisis Hasil Pengujian}
Analisis hasil pengujian berisi analisis terhadap hasil pengujian yang telah dilakukan. Analisis dilakukan dengan membandingkan hasil pengujian dengan spesifikasi kebutuhan yang telah ditetapkan sebelumnya. Apabila hasil pengujian sesuai dengan spesifikasi kebutuhan, maka sistem dapat dikatakan berhasil. Sebaliknya, apabila hasil pengujian tidak sesuai dengan spesifikasi kebutuhan, maka sistem dapat dikatakan gagal.

\section{Pembahasan Hasil Pengujian}
Pembahasan hasil pengujian berisi pembahasan terhadap hasil pengujian yang telah dilakukan. Pembahasan dilakukan dengan membandingkan hasil pengujian dengan spesifikasi kebutuhan yang telah ditetapkan sebelumnya. Apabila hasil pengujian sesuai dengan spesifikasi kebutuhan, maka sistem dapat dikatakan berhasil. Sebaliknya, apabila hasil pengujian tidak sesuai dengan spesifikasi kebutuhan, maka sistem dapat dikatakan gagal.