Perkembangan Internet of Things (IoT) menimbulkan kebutuhan yang semakin meningkat untuk memastikan keamanan komunikasi mesin-ke-mesin (M2M). Dalam konteks FHIR (Fast Healthcare Interoperability Resources), di mana pertukaran data kesehatan elektronik antar sistem dan perangkat medis terjadi, otentikasi M2M menjadi krusial. Penelitian ini bertujuan untuk mengevaluasi risiko yang terkait dengan otentikasi M2M dan mengidentifikasi strategi untuk memitigasi risiko tersebut.

Sebagai solusi, penelitian ini menawarkan pendekatan inovatif dengan menggunakan metode klasifikasi Random Forest untuk mengelompokkan parameter otentikasi. Dengan Random Forest, setiap parameter otentikasi dapat diklasifikasikan dengan akurasi yang tinggi. Pendekatan ini diharapkan dapat meningkatkan keefektifan otentikasi M2M dalam mengatasi risiko akses tidak sah dan serangan penolakan layanan. Output dari metode ini adalah prediksi apakah user akan berhasil atau gagal dalam autentikasi.

Hasil penelitian memberikan kesimpulan bahwa pendekatan klasifikasi Random Forest efektif dalam mengevaluasi risiko terkait otentikasi M2M dalam kasus FHIR. Dibandingkan dengan metode heuristik dan pendekatan Random Forest lainnya, hasilnya menunjukkan bahwa Random Forest memberikan prediksi yang lebih akurat dan dapat diandalkan. Kesimpulan ini memberikan landasan bagi penggunaan metode klasifikasi Random Forest dalam meningkatkan keamanan otentikasi M2M, terutama dalam menghadapi risiko akses tidak sah dan serangan penolakan layanan.

\noindent{Kata kunci} : RBA, Autentikasi, M2M, Random Forest
