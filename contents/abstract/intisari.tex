Otentikasi M2M adalah komponen penting untuk mengamankan komunikasi
FHIR \textit{ (Fast Healthcare Interoperability Resources)}, namun kredensial yang bocor
adalah faktor paling umum yang menyebabkan pelanggaran data. Memastikan
bahwa hanya perangkat resmi yang dapat mengakses dan bertukar data satu sama
lain. Studi ini bertujuan untuk menilai risiko yang terkait dengan otentikasi M2M
dan mengidentifikasi risiko tersebut.
Selain itu studi ini akan menggunakan pendekatan berbasis risiko untuk
mengidentifikasi dan menilai potensi risiko yang terkait dengan otentikasi M2M.
Ini akan melibatkan identifikasi pelaku ancaman potensial, kerentanan, dan dampak
dari serangan yang berhasil. Studi ini juga akan mengevaluasi metode otentikasi
M2M saat ini dan keefektifannya dalam mengurangi risiko yang teridentifikasi.

\noindent{Kata kunci} : RBA, Autentikasi, M2M, Random Forest
