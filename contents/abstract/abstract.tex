
	The development of the Internet of Things (IoT) creates an increasing need to ensure the security of machine-to-machine (M2M) communications. In the context of FHIR (Fast Healthcare Interoperability Resources), where electronic health data exchange between medical systems and devices occurs, M2M authentication becomes crucial. This research aims to evaluate the risks associated with M2M authentication and identify strategies to mitigate these risks.
	
	As a solution, this research offers an innovative approach using the Random Forest classification method to group authentication parameters. With Random Forest, each authentication parameter can be classified with high accuracy. This approach is expected to improve the effectiveness of M2M authentication in addressing the risk of unauthorized access and denial of service attacks. The output of this method is a prediction of whether the user will succeed or fail in authentication.
	
	The research results provide the conclusion that the Random Forest classification approach is effective in evaluating risks related to M2M authentication in FHIR cases. Compared with other heuristic methods and Random Forest approaches, the results show that Random Forest provides more accurate and reliable predictions. This conclusion provides a basis for the use of the Random Forest classification method in improving M2M authentication security, especially in the face of the risk of unauthorized access and denial of service attacks.

\noindent\textbf{Keywords} : RBA, Authentication, M2M, Random Forest